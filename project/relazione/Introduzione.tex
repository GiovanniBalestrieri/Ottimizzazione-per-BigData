%\begin{abstract}
\textbf{Abstract} 
\normalsize
The growth of content on the web has been followed by increasing interest in opinion mining. This field of research relies on accurate recognition of emotion from textual data. There?s been much research in sentiment analysis lately, but it always focuses on the same elements. Sentiment analysis traditionally depends on linguistic corpora, or common sense knowledge bases, to provide extra dimensions of information to the text being analyzed.
Previous research has not yet explored a fully automatic method to evaluate how events associated to certain entities may impact each individual?s sentiment perception. This project presents a method to assign valence ratings to entities, using information from their Wikipedia page, and considering user preferences gathered from the user?s Facebook profile. Furthermore, a new affective lexicon is
compiled entirely from existing corpora, without any intervention from the coders.
%\end{abstract}

\subsection*{Introduzione}
\addcontentsline{toc}{chapter}{Introduzione}
\large

%\begin{figure}[h]
%\centering
%\includegraphics[width=5cm]{imgs/segway.jpg}
%\caption{Segway}
%\label{fig:ponteh_sch3}
%\end{figure}

\noindent Il Segway � un veicolo elettrico a due ruote auto-bilanciante. E' una creazione di Dean Kamen, il quale present� il prototipo del primo modello, lo Human Transporter, il 3 Dicembre 2001. Caratterizzato dall'utilizzo di cinque giroscopi allo stato solido e di un computer interno al fine di mantenere l'equilibrio, il Segway HT non utilizzava freni ed arrivava alla velocit� di circa 19 km orari; tramite la tecnologia che gli sviluppatori chiamarono "Dynamic Stabilization", la quale prevedeva una monitorizzazione della posizione del centro di massa circa 100 volte al secondo, la velocit� ed il verso dello spostamento erano controllate dal modo in cui il guidatore spostava il peso del corpo, mentre la direzione era gestita tramite una manopola posta sul lato sinistro del manubrio. Fu proprio quest'ultima caratteristica ad essere modificata con il modello successivo, il Segway PT (Personal Transporter), nel quale il sistema di sterzata � stato collegato al movimento del piantone del manubrio, il quale non � pi� rigido, ma inclinabile verso i due lati.\\
Scopo di questa tesi � mostrare come � stato creato un primo (non in ordine cronologico, ma in quanto passo iniziale di un progetto pi� ampio) prototipo di segway, andando a descrivere nello specifico come � stata costruita la struttura di base del robot, composta da due motori in corrente continua ed una piattaforma per i componenti elettronici, nel primo capitolo, nel quale si tratter� quindi la parte del lavoro riguardante l'hardware. Il secondo capitolo riguarda invece l'analisi dei motori, tramite lo studio del loro comportamento dal punto di vista analitico (con l'identificazioni delle loro funzioni di trasferimento) e la creazione di relativi sistemi di controllo. Sar� poi demandata al capitolo tre la descrizione di un modello relativo al comportamento generale di un prototipo cos� strutturato, al fine di trovare le relazioni che legano le tensioni applicate ai motori e gli effetti che si ottengono nella movimentaizione del robot. Nel capitolo quattro infine si mostrano i risultati simulativi ottenuti tramite l'utilizzo di un controllore lineare quadratico applicato al modello ottenuto con lo scopo di stabilizzare il segway in posizione verticale. 
