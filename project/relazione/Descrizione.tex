\chapter{The Dc motor model}
\label{chap:Descrizione}
\section{System equations}
The aim of the project is to identify the parameters of the transfer function of a Dc motor.
These motors have wide applications in industrial control systems because they are easy to control and model. For analytical control system design, sometimes a precise model of the DC motor used in a control system may be needed.
\\
A DC motor is a mechanically commutated electric motor powered from direct current (DC). It is composed by 2 main parts: the stator which is stationary in space and by the rotor which is the rotating part. The current in the rotor is switched by the commutator.
DC motors have a rotating armature winding (in which a voltage is induced) but static armature magnetic field and a static field winding permanent magnet. Different connections of the field and armature winding provide different inherent speed/torque regulation characteristics. The speed of a DC motor can be controlled by changing the voltage applied to the armature or by changing the field current. The introduction of variable resistance in the armature circuit or field circuit allowed speed control. 
The electric equivalent circuit of the motor is shown in the following picture.
%
%\begin{figure}[htbp]
%\centering
%\includegraphics[width=0.5\textwidth]{imgs/circuit.jpg}
%\caption{Dc Motor equivalent circuit } \label{fig:Circuit}
%\end{figure}
%
%The rotor and shaft are assumed to be rigid and we will assume a viscous friction model characterized by a friction torque $\varsigma$ proportional to the angular velocity.
%\begin{equation}
%\varsigma=b\dot{\theta}
%\end{equation}
%The torque generated by a DC motor is proportional to the armature current and the strength of the magnetic field. In this paper we will assume that the magnetic field is constant and, therefore, the motor produces the following torque, which is proportional to the current through the motor windings \textit{i} by a constant factor $K_{t}$ as shown in the equation below. 
%\begin{equation}
%T=K_{t}i
%\end{equation}
%The back emf, \textit{e}, is proportional to the angular velocity $\dot{\theta}$ of the shaft by a constant factor $K_{e}$. The permanent magnets in the motor induce the following back emf in the armature:
%\begin{equation}
%e=K_{e}\dot{\theta}
%\end{equation}
%We will assume that there are no electromagnetic losses. This means that mechanical power is equal to the electrical power dissipated by the back emf in the armature. Therefore the motor torque and back emf constants are equal: $K_{t}=K_{e}=K$.
%\\
%The equations of motion for Dc motors are as follows:
%\begin{align}
%V&=Ri+L\frac{di}{dt}+K_{e}\dot{\theta}\\
%J\ddot{\theta}&=K_{t}i-b\dot{\theta}-\tau 
%\end{align}
%where J is the rotor's moment of inertia, R the motor winding resistance, b the motor's viscous friction constant, $\tau$ the torque applied to the rotor by an external load and L the motor inductance.
%Applying the Laplace transformation, the above modelling equations can be expressed in the Laplace variable s:
%\begin{align}
%V(s)&=Ri(s)+sLi(s)+sK_{e}\theta(s)\\
%s^{2}J\theta(s)&=K_{t}i(s)-sb\theta(s)\\
%(Ls+R)i(s)&=V(s)-Ks\theta(s) \label{eq:1}\\
%Ki(s)&=s(Js+b)\theta(s) \label{eq:2}
%\end{align}
%The above equations yield the open-loop transfer function by eliminating \textit{i(s)} between (\ref{eq:1}) and (\ref{eq:2}), where the rotational speed is considered the output and the armature voltage is considered the input.
%\begin{equation}
%P(s)=\frac{\omega(s)}{V(s)}=\frac{K}{(Ls+R)(Js+b)+K^{2}} \label{modelG}
%\end{equation}
%Considering the electric current through the rotor windings and the shaft's angular velocity as state variables, it is possible to express the system in the state-space representation:
%\begin{equation}
%\left[ \begin{matrix}
%\ddot{\theta} \\
% \dot{I}
%\end{matrix}\right] = \left[ \begin{matrix}
%-\frac{b}{J} & \frac{K}{J} \\
% -\frac{K}{L} & -\frac{R}{L}
%\end{matrix}\right] \left[ \begin{matrix}
%\dot{\theta} \\ I
%\end{matrix}\right] + \left[ \begin{matrix}
%0 \\
% \frac{V}{L}
%\end{matrix}\right]
%\end{equation}
%\begin{equation}
%y=\left[ 1  0 \right] \left[ \begin{matrix} 
%\dot{\theta} \\ 
%I
%\end{matrix}\right]
%\end{equation}
%
\section{Simulink implementation}

The Dc motor system has been implemented in Simulink by summing the torques acting on the rotor inertia and integrating the angular acceleration to give angular velocity.  Figure \ref{fig:SimMod} shows the implemented model from equations derived in the first section.
%
%\begin{figure}[htbp]
%\centering
%\includegraphics[width=0.8\textwidth]{imgs/DcMotorSimulink.png}
%\caption{Simulink implementation of the dc Motor block} \label{fig:SimMod}
%\end{figure}
%
%We will then consider the voltage V as input and the angular velocity of the shaft $\omega$ as the output. Let's suppose we want the shaft to rotate at a desired angular velocity $\omega_{r}$ in order to implement a speed control law. In open loop configuration the system's response to a step input of amplitude 1V is shown below:
%
% \begin{figure}[H]
%\centering
%\includegraphics[width=0.8\textwidth]{imgs/StepRes.png}
%\caption{Open loop step response} \label{fig:OLR}
%\end{figure}
%
%As we can see from Figure 1.3 with an input of amplitude $V = 1 V$, the ouput is $\omega = 0.1 rad/s$. This behaviour is due to the physical laws we've introduced so far, the response changes as a function of the values of the paramenters appearing in the transfer function(\ref{modelG}). 
%
%\begin{equation}
%P(s)=\frac{\omega(s)}{V(s)}=\frac{1273}{(s+10)(s+220)}
%\label{MotorTf}
%\end{equation}   
%
%P(s) establishes the relation between the input provided to the system and the resulting ouput considered. Each motor will have a proprer behaviour, its output can be viewed as an expression of its internal dynamics. For example the transfer function \ref{MotorTf} yields the input response shown in Figure \ref{fig:OLRes}.  
%
% \begin{figure}[htbp]
%\centering
%\includegraphics[width=1\textwidth]{imgs/Step2.png}
%\caption{Open loop step response with input amplitude 3V and 2sec delay} \label{fig:OLRes}
%\end{figure}

\section{The objectif}

Let's consider the problem of tracking a desired output value, in this case $\omega$. This requires a few sensors in order to compare values of the signals provided as input and to get output informations such as the shaft's angular velocity. 

%We need to determine the transfer function of the system we want to control. The knowledge of the ouput-input relation is compulsory to establish a control law. The first step is the one in which the motor's paramenters are determined in order to compute the output of the system. The motor behaviour is a function of the value inertia's moment J, the rotor winding resistance R, the viscous friction constant b, usually neglected in this kind of application, the torque applied $\tau$ and the inductance L. These values are not always provided by the manufacturer, so they need to be measured or estimated. 
%In the next chapter the Least Squares algorithm will be discussed, this approach proposes a parameter update law in which the integral squared error is minimized to estimate the nominal values of the parameters in (\ref{modelG}).
